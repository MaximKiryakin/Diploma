%основное
\usepackage{cmap} % для кодировки шрифтов в pdf
\defaulthyphenchar=127 
\usepackage[T1,T2A]{fontenc} 
\usepackage[utf8]{inputenc}
\usepackage[english,russian]{babel}

%\usepackage{indentfirst}\parindent=2.5em 

%математика
\usepackage{amsmath,amstext,amsfonts,amssymb}
\usepackage{mathtext,mathrsfs,array}

\usepackage{moreverb}

%% Выбор шрифтов %% 

%\usepackage{mathptmx}
%\usepackage{helvet} %arev, avant, bookman, chancery, charter, euler, helvet, lmodern,mathtime ,mathptm, mathptmx, newcent, palatino, pifont, utopia.
%\usepackage{bookman}

%\usefonttheme{serif} % нормальные шрифты текста-формул в целом
\usefonttheme{professionalfonts} % все sffamily кроме формул (они serif)
%\usepackage{concmath}
%\usepackage{cmbright}
%\usefonttheme[stillsansserifmath]{serif}
%\usefonttheme[onlymath]{serif} % Использовать шрифт с засечками для формул (+ \usepackage{cmbright} --- хорого отображаются гречиские буквы)
%\usefonttheme[onlylarge]{structurebold} % main , frame titles, and section entries in the table of contents будут жирненькими
%\setbeamerfont{institute}{size=\normalsize} % Более крупный шрифт для подзаголовков титульного листа

%\setbeamerfont{title}{family=\sffamily}%,size=\large
%\setbeamerfont{subtitle}{family=\sffamily,size=\small}
\setbeamerfont{author}{size=\normalsize}
%\setbeamerfont{institute}{size=\normalsize}
\setbeamerfont{date}{size=\normalsize}
%\setbeamerfont{frametitle}{family=\sffamily}
%\setbeamerfont{block title}{family=\sffamily,size=\large}
%\setbeamerfont{contents}{family=\sffamily}

%\setbeamerfont{bibliography item}{size=\small}
%\setbeamerfont{bibliography entry author}{shape=\itshape}
%\setbeamerfont{bibliography entry title}{size=\small}
%\setbeamerfont{bibliography entry location}{size=\small}
%\setbeamerfont{bibliography entry note}{size=\small}

\useinnertheme{circles}
%\useinnertheme{rectangles}


\usepackage{soulutf8}% Поддержка переносоустойчивых подчёркиваний и зачёркиваний
\usepackage{icomma}  % Запятая в десятичных дробях

%%%%%%%%%%%%%%%%%%%

% Если используется последовательное появление пунктов списков на
% слайде (не злоупотребляйте в слайдах для защиты дипломной работы),
% чтобы еще непоявившиеся пункты были все-таки немножко видны.
\setbeamercovered{transparent}

% отключить клавиши навигации
\setbeamertemplate{navigation symbols}{}

%command to set the logo to nothing
\newcommand{\nologo}{\setbeamertemplate{logo}{}} 


\usepackage{comment}  %для комментирования большого крол-ва строк

\newcommand{\specialcell}[2][c]{% объявляем новую команду для переноса строки внутри ячейки таблицы
	\begin{tabular}[#1]{@{}c@{}}#2\end{tabular}}

%гафика 
\usepackage{pgf,tikz,pgfplots}
\pgfplotsset{compat=1.15}
\usepackage{tcolorbox}
\usetikzlibrary{arrows,arrows.meta,patterns,shadings,shadows}
\tcbuselibrary{skins,breakable}